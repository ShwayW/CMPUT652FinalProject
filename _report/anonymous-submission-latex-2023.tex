%File: anonymous-submission-latex-2023.tex
\documentclass[letterpaper]{article} % DO NOT CHANGE THIS
\usepackage[submission]{aaai23}  % DO NOT CHANGE THIS
\usepackage{times}  % DO NOT CHANGE THIS
\usepackage{helvet}  % DO NOT CHANGE THIS
\usepackage{courier}  % DO NOT CHANGE THIS
\usepackage[hyphens]{url}  % DO NOT CHANGE THIS
\usepackage{graphicx} % DO NOT CHANGE THIS
\urlstyle{rm} % DO NOT CHANGE THIS
\def\UrlFont{\rm}  % DO NOT CHANGE THIS
\usepackage{natbib}  % DO NOT CHANGE THIS AND DO NOT ADD ANY OPTIONS TO IT
\usepackage{caption} % DO NOT CHANGE THIS AND DO NOT ADD ANY OPTIONS TO IT
\frenchspacing  % DO NOT CHANGE THIS
\setlength{\pdfpagewidth}{8.5in} % DO NOT CHANGE THIS
\setlength{\pdfpageheight}{11in} % DO NOT CHANGE THIS
%
% These are recommended to typeset algorithms but not required. See the subsubsection on algorithms. Remove them if you don't have algorithms in your paper.
\usepackage{algorithm}
\usepackage{algorithmic}

%
% These are are recommended to typeset listings but not required. See the subsubsection on listing. Remove this block if you don't have listings in your paper.
\usepackage{newfloat}
\usepackage{listings}
\DeclareCaptionStyle{ruled}{labelfont=normalfont,labelsep=colon,strut=off} % DO NOT CHANGE THIS
\lstset{%
	basicstyle={\footnotesize\ttfamily},% footnotesize acceptable for monospace
	numbers=left,numberstyle=\footnotesize,xleftmargin=2em,% show line numbers, remove this entire line if you don't want the numbers.
	aboveskip=0pt,belowskip=0pt,%
	showstringspaces=false,tabsize=2,breaklines=true}
\floatstyle{ruled}
\newfloat{listing}{tb}{lst}{}
\floatname{listing}{Listing}
%
% Keep the \pdfinfo as shown here. There's no need
% for you to add the /Title and /Author tags.
\pdfinfo{
/TemplateVersion (2023.1)
}

% DISALLOWED PACKAGES
% \usepackage{authblk} -- This package is specifically forbidden
% \usepackage{balance} -- This package is specifically forbidden
% \usepackage{color (if used in text)
% \usepackage{CJK} -- This package is specifically forbidden
% \usepackage{float} -- This package is specifically forbidden
% \usepackage{flushend} -- This package is specifically forbidden
% \usepackage{fontenc} -- This package is specifically forbidden
% \usepackage{fullpage} -- This package is specifically forbidden
% \usepackage{geometry} -- This package is specifically forbidden
% \usepackage{grffile} -- This package is specifically forbidden
% \usepackage{hyperref} -- This package is specifically forbidden
% \usepackage{navigator} -- This package is specifically forbidden
% (or any other package that embeds links such as navigator or hyperref)
% \indentfirst} -- This package is specifically forbidden
% \layout} -- This package is specifically forbidden
% \multicol} -- This package is specifically forbidden
% \nameref} -- This package is specifically forbidden
% \usepackage{savetrees} -- This package is specifically forbidden
% \usepackage{setspace} -- This package is specifically forbidden
% \usepackage{stfloats} -- This package is specifically forbidden
% \usepackage{tabu} -- This package is specifically forbidden
% \usepackage{titlesec} -- This package is specifically forbidden
% \usepackage{tocbibind} -- This package is specifically forbidden
% \usepackage{ulem} -- This package is specifically forbidden
% \usepackage{wrapfig} -- This package is specifically forbidden
% DISALLOWED COMMANDS
% \nocopyright -- Your paper will not be published if you use this command
% \addtolength -- This command may not be used
% \balance -- This command may not be used
% \baselinestretch -- Your paper will not be published if you use this command
% \clearpage -- No page breaks of any kind may be used for the final version of your paper
% \columnsep -- This command may not be used
% \newpage -- No page breaks of any kind may be used for the final version of your paper
% \pagebreak -- No page breaks of any kind may be used for the final version of your paperr
% \pagestyle -- This command may not be used
% \tiny -- This is not an acceptable font size.
% \vspace{- -- No negative value may be used in proximity of a caption, figure, table, section, subsection, subsubsection, or reference
% \vskip{- -- No negative value may be used to alter spacing above or below a caption, figure, table, section, subsection, subsubsection, or reference

\setcounter{secnumdepth}{0} %May be changed to 1 or 2 if section numbers are desired.

% The file aaai23.sty is the style file for AAAI Press
% proceedings, working notes, and technical reports.
%

% Title

% Your title must be in mixed case, not sentence case.
% That means all verbs (including short verbs like be, is, using,and go),
% nouns, adverbs, adjectives should be capitalized, including both words in hyphenated terms, while
% articles, conjunctions, and prepositions are lower case unless they
% directly follow a colon or long dash
\title{Learning Play Styles on Super Mario Bros: \\ Generating Levels by Augmenting Human Data via Imitation Learning}
\author{
    %Authors
    % All authors must be in the same font size and format.
    Michael Guevarra\equalcontrib,
    Justin Stevens\equalcontrib,
    Shuwei Wang\equalcontrib
}
\affiliations{
    %Afiliations
    \textsuperscript{\rm 1}Association for the Advancement of Artificial Intelligence\\
    % If you have multiple authors and multiple affiliations
    % use superscripts in text and roman font to identify them.
    % For example,

    % Sunil Issar, \textsuperscript{\rm 2}
    % J. Scott Penberthy, \textsuperscript{\rm 3}
    % George Ferguson,\textsuperscript{\rm 4}
    % Hans Guesgen, \textsuperscript{\rm 5}.
    % Note that the comma should be placed BEFORE the superscript for optimum readability

    1900 Embarcadero Road, Suite 101\\
    Palo Alto, California 94303-3310 USA\\
    % email address must be in roman text type, not monospace or sans serif
    publications23@aaai.org
%
% See more examples next
}




% REMOVE THIS: bibentry
% This is only needed to show inline citations in the guidelines document. You should not need it and can safely delete it.
\usepackage{bibentry}
% END REMOVE bibentry

\begin{document}

\maketitle

\begin{abstract}
AAAI creates proceedings, working notes, and technical reports directly from electronic source furnished by the authors. To ensure that all papers in the publication have a uniform appearance, authors must adhere to the following instructions.
\end{abstract}

\section{Introduction}
Procedural Content Generation (PCG) has been used a lot in recent years to generate content for games. One popular game is Mario, which has a plethora of papers \cite{dahlskog2014multi, summerville2016super} and even consists of an entire framework for applying AI to the game \cite{karakovskiy2012mario}. One open research area in applying PCG is that of generating content that is tailored to players through personalization \cite{yannakakis2015experience} \cite{summerville2016learning}. In personalization, a model of the players is learned based on how the player engages with the game \cite{yannakakis2013player}. 

One limitation of player modeling is it's very cumbersome to have users automatically play through levels and give lots of examples of their desired playstyle. However, many PCG approaches that use machine learning are very data hungry and require lots of data to successfully train. Therefore, it'd be ideal if we could augment the data that a player provides by using imitation learning to imitate a user's playstyle to provide more data to a PCG system \cite{hussein2017imitation}. In doing so, we'd be able to have a PCG system that can more easily adapt to a wide variety of playstyles and also use more powerful PCG systems that are data-hungry such as transformers \cite{vaswani2017attention}.

In our work, we offer several novel contributions to generating personalized content for players:
\begin{itemize}
    \item Integrating the Mario-AI framework with OpenAI gym to extract player paths and metadata from a human or agent's playthrough of a Mario level. 
    \item Experiments demonstrating that imitation learning agents can generate similar player paths to a given user capturing a distinct play style. 
    \item Applying transformers to generate Mario levels. 
    \item An evaluation framework for evaluating whether generated levels in a game match a player's playstyle. 
\end{itemize}




\section{Related Work}



- NATURE paper using RL to solve atari games\\
... we just copy the CNN + frame stacking architecture\\
- maybe a survey on different Learning from Demonstration (LfD) techniques (BC, GAIL, AIRL)\\
- imitation learning in video games (link that matthew sent us in his first critique)\\

\section{Methodology}

Everyone to work on the parts they contributed to

% \subsection{Environment Setup} 
\subsection{Mario-AI Framework as a Gym Environment}

- converting Mario-AI framework into a gym environment\\
    ... explain benefits (custom levels, a* agents, fwd model + game model, game stats, privileged information like invisible tiles etc.)\\
    ... implementation details\\

- state representation\\
    ... [figure] to visualize 11x11 window around mario + frame stacking (nature paper)\\
    ... [figure or table] to show mapping of tile number to grayscale\\
    
- alternative design choices / experiments / justifications\\
    ... fwd model vs game model (fwd better on ordinary RL but worse when doing LfD)\\
    ... Mario-centred vs screen centred\\
    ... 11x11 window vs other window sizes\\
    ... alternative CNN architectures (1-hot encoding, changes to layers/kernels)
    ... Frame stacking/skipping (skipping good in ordinary RL but very bad in LfD, tried to apply rule-based stuff on skipped actions but did not work well, too much data loss)
    ... random starts (location)\\
    ... sticky actions (stochasticity) *** Explain how mario is very deterministic (pros and cons, maybe allude to this in discussion instead), tried random actions but diluted the source playstyle data\\
    ... fixed horizons (blit screen with a pixel vs keep last obs to show good/bad of dying/completing early)\\
    
\subsection{Imitation Learning}
- Trajectory collection process (can handle user input or agent input), + preprocessing details\\
- *** At the same time we could throw in implementation details of path collection that happens at this same step\\
- stable baselines 3\\
- BC, GAIL, AIRL\\
- Q: should subsectioning the levels go here or in evaluation?

\subsection{The Decoder-only Transformer Level Generator}


\section{Experiments}
Not sure if we want this as its own section or a subsection inside either methodology or evaluation but for sure need a dedicated area to outline our various experiments, we can label each experiment so we can refer to it easier in the rest of the paper
\begin{itemize}
    \item Experiment 1 evaluating generator percent completable levels (random vs biased prompts)
    \item Experiment 2: evaluating generator plagiarism (completionist vs speedrunner)
    \item Experiment 3: comparison(s) between completionist and speedrunner outputs
    \item Experiment 4: BC vs GAIL vs AIRL
    \item Experiment 5: BC - testing agent on new levels (new subsections not levels)
    \item Experiment 6: BC - path variations from 1 sample
\end{itemize}

\subsection{Experiment 1}
- explain


\section{Evaluation}
One of the goals of our project was to see if the generator is adapting to the individual user's playstyles. 

- establish why edit distance is a good indicator and why we use it throughout everything\\
- Explain how we evaluate each playstyle?\\
... Speedrunner\\
... Completionist\\
... Enemy-stomper (lets go with percent of total enemys killed in the level + full completion)\\

\section{Results}
Everyone to work on the parts they contributed to 

\subsection{Imitation Learning}
- experiment 4: BC vs GAIL vs AIRL on lvl-1 (subsections)\\
... EXPLAIN the setup and why (3 different types of models, help break problem down, do not affect input to transformer negatively, can also act as a proxy of being 3 different levels, Edit distance for consistency) \\
... [figure] Can still present a training curve of the different algorithms if we want\\
... [figure] of paths similar to in final presentation\\

- experiment 5: BC with 5-samples (or 1 only depending on results), matrix running agents on level subsections not trained on (i.e. matrix that shows lvl-11 agent on lvl-11, lvl-12, lvl-13, etc.) -- glimpse at agent running on 'new' level not seen without us biasing by creating the new level ourselves\\
... [table] showing above results in a matrix\\

- experiment 6: BC agent with 1 sample, providing more than 1 sample on the same and on new level. Maybe test if adding sticky action can improve variation (stdev) but retain goals\\
... Although we do not fully integrate path to our generator, we are exploring how we can provide variations of path data that still retain underlying playstyle!\\
- [figure] visualizing more than 1 path on a level to show variation\\
- [table] next to it with tabular results (avg edit distance +/- stdev)\\

\section{Discussion}
Primarily Michael 

- Summarize out takeaways and restate them in the context of a class deliverable. Mainly talk about our limited time and what we were able to accomplish vs what we wanted to accomplish\\
- most 'discussion' likely happens in result section\\
- other noteworthy 'cool' stuff we noticed that arent a takeaway\\

\subsection{Future Work}
- should spend lots of time here explaining what our current progress could lead towards, and the prospective benefits/novelty of future work\\

- tuning of RL algorithms (ability of AIRL to generalize more, recover the learned reward function,)\\
- revisit 1-hot encoding state space + other implementation optimizations \\

\section{Conclusion}
Primarily Shway 

\section{Group Member Breakdown}

\bibliography{aaai23.bib}
\end{document}

